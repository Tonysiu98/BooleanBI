\documentclass{report}   	
\usepackage{geometry}                		% See geometry.pdf to learn the layout options. There are lots.



\title{Project Plan \\ Formalising The Boolean BI Display Calculus in Isabelle/HOL}
\author{Chi Cheong Tony Siu \\ \\ 3rd Year Dissertation\\ 
BSc Computer Science\\ \\
Supervisor: James Brotherston}
\date{1st November 2019}							% Activate to display a given date or no date


\begin{document}
\maketitle
%\section{}
%\subsection{}
\setcounter{page}{1}

\section*{1. Aims and Objectives}

\Large \textbf {Aims:} To formalise a display calculus proof system for the bunched logic Boolean BI in the proof assistant Isabelle/HOL, and to prove that system sound and complete with respect to its Hilbert axiomatisation. \\ \\
\textbf {Objectives:}  1. Review ' A unified Display Proof Theory for Bunched Logic', in particular to Boolean BI's definition and proofs, and understand the mechanism of the proof assistant Isabelle/HOL. 2. Develop the syntax for Boolean BI in Isabelle. i.e. the definition of  BBI formula, Structure Consecution and proof system syntax. 3. Formalise soundness and completeness of the system using Isabelle automatic tools.

\section*{2. Deliverables} 
\begin{enumerate}
	\item Formalised BBI proof system system
		\begin{itemize}
    			 \item definition of BBI formula
			 \item definition of formula axioms and rules in Hilbert axiomatisation
			 \item definition of Structures (Antecedent and Consequent)
			 \item definition of Consecution validity
			 \item definition of Display Calculi for BBI
		\end{itemize}
	\item Formalised Soundness and Completeness of the logic system
		\begin{itemize}
			\item all lemmas are proved in Isabelle
			\item all theorems are proved in Isabelle
		\end{itemize}
	\item Complete Dissertation using Latex
\end{enumerate}

\newpage
\section*{3. Work Plan}
\begin{itemize}
	\item Over the Summer : Literature Review and learn Isabelle's syntax and proof mechanism
	\item Project Start to End of October (6 weeks) :  Enhance on Isabelle development skills and formalised basic definition of BBI formula and Antecedent and Consequent Structures with proof system axioms and rules.
	\item Start of November to End of Term 1 (6 weeks) : Finishes all definitions such as Display Calculi for BBI and start the development of Soundness proof. In the same, writes up the current progress in Latex.
	\item End of Term 1 to Mid March (16 weeks) : Formalised Both Soundness and Completeness proofs, including Cut elimination Theorem
	\item Mid March onwards(4 weeks) : Work on the final report
\end{itemize}
\end{document}  